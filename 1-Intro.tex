\section{Introduction}

In the today's fast-paced world free access to means of communication and information should be considered a fundamental human right.
Frighteningly, in the majority of cases this right is very far from being truely granted.
Most of the communications' channels are either state- or industry-owned and governed, putting exchange of knowledge and news at the mercy of these players/actors.
Freedom of speech, the work of human rights advocates and privacy of communication depend on the good will of the infrastructure owners (syn).

This alarming state of affairs has been rightly noted/assessed and in the past couple of decades, a number of grassroot projects has arisen which try to counteract the situation.
The free/community network movements have set themselves the goal of building a communication infrastructure owned and controlled by its users.
Understanding why activists engage in these projects can give them key insights into the inner workings of the corresponding projects and help them building a sustainable community, allow them to improve their processes, and encourage new people to participate.

However, till that moment there isn't much academic research/work conducted on the topic of community networks and scarcely any which investigates the reasons why people partake of them.
Parting from research/investigation on a seemingly related topic: the motivation of free and open source software developers, the present paper aims to outline some first impressions on the motivation of community network activists.

\begin{comment}
* Untersuchungsgegenstand
* Erkenntnisinteresse
* Forschungsstand
* Ergebnisse können/sollen angedeuten werden

Why is the topic relevant??

* grassroots movements
* one of the free/open movements which is relatively understudied till now
* understanding why
  ** may enhance motivation and performance (or not);
  ** or give activists insights into how to improve their processes
\end{comment}


