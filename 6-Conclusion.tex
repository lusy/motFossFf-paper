\section{Conclusion}

The present paper discussed some first impressions on the possible motivations which drive community network activists of the Germany based project Freifunk.
Although the sample of interviewees was fairly small (16 people) and the employed methodology may lack some scientific thoroughness, some interesting patterns have emerged.
%patterns
Ideology was one of the primary motivations for the interviewed activists, despite the curious fact that several of them viewed the possibility of truly fulfilling their ideas as utopian.
We observed idiosyncrasies in the motivations to join in the different generations Freifunkers:
people who joined in the early days of the project were mostly driven by personal need, a factor which seems to play a rather minor role today.
We also noted the tendency of people's motivations evolving over time,
which together with the diversity of tasks which the Freifunk project comprises may play a crucial role for maintaining long-term contributors.
Further investigations with improved methods and including a larger participants' sample from different community network projects would definitely be insightful, since every project has its own specific characteristics and priorities.


\begin{comment}
* an investigation of further community network projects would definitely be insightful, since all of them have sometimes quite different aspects, ideas, definitions, motivations

* not clear whether such a comparison is really valid: methods for gathering results in both domains vary widely

* Die Papers beschäftigen sich (vor allem) mit der Motivation von Entwickler*innen. Allerdings gibt es auch andere Möglichkeiten zu beiden Communities beizutragen (user documentation, publicity, design, Übersetzung, ...). Kann man sich da andere Motivationen vorstellen?

* critical discussion of the own methodology
\end{comment}
