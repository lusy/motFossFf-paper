\section{Motivations in the Freifunk Community}
%TODO: Rename section to "Findings"?

general part of the questionnaire
--> cluster answers

some graphs/tables on the specific questions

\begin{comment}
* is the focus of the paper;

* only pull up the foss motivations as comparison to underline similarities and differences with the foundings here

* find a suitable neat form for presenting the foundings
\end{comment}

\begin{comment}
Demographics:

gender:
* 2f
* 14m

hintergrund:
* alle: technische Berufe/Informatik;
* 3 Menschen studieren noch;
* 5 arbeiten auf dem Gebiet Netzwerke (eine Person ist nach/über Freifunk dazu gekommen)

wie lange dabei
* in den letzten 2 Jahren angefangen: 7 Personen
* seit ~3 Jahren: 5 Personen
* 1 Person: 2004-2008
* $> 8$ Jahre: 3 Personen

community

* Berlin: 9
* Rheinland: 3
* Hannover: 1
* Bremen: 1
* Bielefeld: 1
* Hamburg: 1
\end{comment}

\begin{comment}
# Die Konkreten Fragen

### Freifunk ist technisch spannend/anspruchsvoll und es macht Spaß, sich damit auseinander zu setzen

* 10 ja
* 1 nicht soo anspruchsvoll aber schon spannend
* 1 joa, nicht Primärmotivation: technisch anspruchsvoll, aber nicht immer spannend^^;
* 1 technisch eher langweilig

---

### Ich will mir die technischen Fähigkeiten aneingnen, wie WLAN/Mesh-Netzwerke/... funktionieren

* 6 ja
* 1 muss man zwangsläufig, um die idee umzusetzen
* 1 nicht mit der Absicht hingegangen aber kam später dazu; "xy läuft nicht, ich wills fixen"
* 2 ja, einzelne Themen, nicht Primärfokus
* 3 Vorwissen mitgebracht/wusste schon wie alles geht im Grunde

---

### Ich will mich mit den gesetzlichen Rahmenbedinungen auseinandersetzen

* 9 naja, muss man zwangsläufig, wenn man das Projekt betreiben möchte (spannend unklar, aber wichtig)
* 2 nee eher nicht
* 1 nee, will sich nicht in vorgegebenen hierarchischen Strukturen einfügen
* 1 keine Einstiegsmotivation, aber spannend sich damit auseinander zu setzen
* 1 ist nicht uninteressant, sollten sich aber lieber Menschen mit beschäftigen, die das besser können (die juristische Sprache verstehen?)
* 1 ja (im Lobbyumfeld unterwegs, politische Motivations dabei; nicht Hauptmotivation, aber kam später dazu)

---

### Ich will eine freie, community-betriebene Kommunikationsinfrastruktur (mit)schaffen, weil ich denke, es ist Scheiße, dass alles in den Händen von paar Konzernen ist

* 8 ja (3 davon sehr stark bejaht :))
* 2 ja (zivilgesellschaftliche Kommunikationsinfrastruktur; die Industrie hat eine eigene; der Staat auch)
* 3 ja (aber glaub nicht mehr, dass das Freifunk komplett bringen kann)
* 1 ja (ist schwierig wenn einer Person viel von der dezentralen Infrastruktur gehört)

---

### Ich hab damit angefangen, weil die kommerziellen Providern keine Lust hatten, mich anzuschließen, weil es sich für sie nicht gelohnt hat

* 2 ja
* 10 nein
* 1 nein, aber demnächst an einem geflüchtetenprojekt mitarbeiten, wo das durchaus motivation ist
* 1 nein, stellt anderen den eigenen Anschluss zur Verfügung

---

### Mir ist Netzneutralität wichtig

* 6 ja
* 1 ja (sesibilität dafür im Freifunk kontext ist nicht unbedingt gegeben)
* 4 ja (aber sieht FF nicht als Kampf für Netzneutralität)
* 1 ja (aber die um jeden preis zu fordern ist schwierig und nicht sehr schlau)
* 1 ja (Pico-Peering-Agreement ist total wichtig!)
* 1 ja (man kanns zumindest versuchen :))

---

### Ich mach das mit Freifunk beruflich: krieg Geld für Knoten einrichten, Wartung, ...

* 9 nein
* 1 joa, kann man als Beruf sehen^^ steckt ziemlich viel Zeit und Energie rein und hat schonmal im Kontext Geld für die eigene Arbeit bekommen
* 1 macht jetzt was mit WiFi und Mesh beruflich (nicht mit Freifunk); eher Konsequenz aus Freifunk als Motivation dafür
* 1 macht beruflich was mit wlan installationen, versuchts aber von ff zu trennen

---

### Ich mach/will machen was auf dem Gebiet beruflich und das ist die Gelegenheit, meine Fähigkeiten aufzupollieren/Kontakte knüpfen/für Arbeitgeber_innen sichtbar werden

* 2 nein (eher: war der Bereich, wo ich meine bereits vorhandenen Kenntnisse anwenden konnte)
* 6 nein
* 1 nein (aber wenn was mit netzwerk machen wollen würde, würde die ff community schon als Kommunikations-/Kontaktplattform sehen)
* 3 nein (passiert sicher, ist aber keine motivation sich mit dem Projekt zu beschäftigen)
* 2 nein (schreibt man schon in den Lebenslauf aber keine Motivation um dahinzugehen)
\end{comment}

Although net neutrality is one of the key principles in the Pico-Peering-Agreement, the minimal consensus paper/document(?) for all Freifunk communities/free network projects, not all of the survey participants interpreted the notion the same way, nor did they grant it equal importance.

The initial motivations for joining the project for the different generations \textit{Freifunkers} become visible:
for people who joined before 2008 one of the main concerns was lack of fast internet connections in their area of living.
In contrast, those who started contributing after that were mainly (syn!) motivated by interest in the technical aspects of the project or its political idea/aspirations.

Several participants mentioned the community aspect of their work as a driving force.
One talked about ``building a project together with others'', ``collaborating with and getting to know people of different ages and backgrounds, which would have hardly happened in a different setting'', ``expanding one's horizons and getting out of one's comfort zone''.
%TODO: footnote: translations are mine, since the interviews were conducted in German
We can recognise here the ``relatedness to others'' component from the motivations' research which apparently drives individuals to internalising the activity they engage into.
%TODO: compare with foss

Curiously, only two(?) participants talked about sharing knowledge and empowering others to build their own infrastructure, although these are among the central goals sketched by the community in their self-conception\cite{ffweb}.

The possible motivations were summarised by one of the participants in the following manner: ``there are the ego people and there are the altruists''.

On the whole, we note following similarities and contrasts among motivations in the Freifunk Community compared to FOSS
%TODO: Probably split this between the lines to acompany every observation

\begin{comment}
  ## Spannend (Manche zusammengefasste Beobachtungen):

* Nicht alle verstehen das selbe unter Netzneutralität und nicht alle erachten das als wichtig (je nach Verständnis)
* (initielle) Mitmachmotivationen verschiedener Generationen werden sichtbar:
  ** Menschen, die vor 2008 angefangen haben: Probleme mit (schnellem) Internetzugang bekommen, man nimmt das in die eigenen Hände
  ** Menschen, die später angefangen haben: eher aus Interesse an Netzwerktechnik/ finden die polit. Idee gut
* mehrere sprechen den Community-Aspect an: also die Möglichkeit gemeinsam ein Projekt aufzubauen, interessante Menschen kennen zu lernen, "aus dem eigenen Dunst rauskommen"
* nicht viele Menschen sprechen die Idee an, das eigene Wissen zu teilen (auch wenn es eine der Zentralthemen in der Selbstdarstellung vom Projekt ist)
* "es gibt die Ego-Leute und es gibt die Altruisten"


##  Gemeinsamkeiten/Unterschiede in den Motivationen

* niemand macht Freifunk, weil er_sie dafür bezahlt wird
* (fast) niemand nimmt am Projekt teil, weil er_sie sich davon bessere Kontakte/Jobchancen verspricht (wird eher als Nebenprodukt gesehen) (an FOSS dagegen angeblich schon)
* sowohl bei FOSS als auch bei Freifunk gibt es Menschen, die aus persönlichen Bedürfnissen teilnehmen ("Ich hatte kein Netz" / "Ich brauchte das Feature XY")
* auch wenn mir unklar ist, wie ich Altruismus messen sollte, scheinen bei Freifunk die Mehrheit der Menschen von der politischen Idee einer dezentralen autonomen Kommunikationsinfrastruktur bewegt zu werden
* Communitybildung/-zugehörigkeit wichtig
* ein Großteil der Menschen findet die Beschäftigung mit Netzwerken spannend
* einige wollen ihre Netzwerkkenntnisse und -fähigkeiten ausbauen, scheint aber nicht so in Fokus zu sein (dagegen bei FOSS eher)

---

Die Papers beschäftigen sich (vor allem) mit der Motivation von Entwickler*innen. Allerdings gibt es auch andere Möglichkeiten zu beiden Communities beizutragen (user documentation, publicity, design, Übersetzung, ...). Kann man sich da andere Motivationen vorstellen?
\end{comment}

