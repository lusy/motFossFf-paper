\section{Motivations in the Freifunk Community}
%TODO: Rename section to "Findings"?

\begin{itemize}
  \item is the focus of the paper;
  \item only pull up the foss motivations as comparison to underline similarities and differences with the foundings here
  \item general part of the questionnaire $\rightarrow$ cluster answers
  \item some graphs/tables on the specific questions
\end{itemize}

Leaning on the extrinsic--intrinsic motivation continuum outlined in section XX the reasons for the interviewees to engage in the Freifunk project can be loosely organised in the following clusters:
intrinsic motivation -- it's fun to climb roofs and get to see the city from above; to help develop the firmware; ...
extrinsic motivation: political ideology
extrinsic motivation: feeling part of a community/community obligation
extrinsic motivation: polish skills, networking


%TODO: does it make sense to split along general/specific part of the questionnaire? make a note on that if that's the preferred organisation method

%------------------------
%---general part
%------------------------

% feeling part of a community
Several participants mentioned the community aspect of their work as a driving force.
They talked about ``building a project together with others''\footnote{The interviews were conducted in German. Here mentioned citations have been translated by the author. The author carries responsibility for any inaccuracies.}
, ``collaborating with and getting to know people of different ages and backgrounds, which would have hardly happened in a different setting'', ``expanding one's horizons and getting out of one's comfort zone''.
We can recognise here the ``relatedness to others'' component from the motivations' research which apparently drives individuals to internalising the activity they engage into.
%TODO: compare with foss

%sharing knowledge
Curiously, only two(?) participants talked about sharing knowledge and empowering others to build their own infrastructure, although these are among the central goals sketched by the community in their self-conception\cite{ffweb}.


%----------------------
%---specific part
%-----------------------

%technical challenge/fun
The majority/a big percentage/the greater part/... of the participants seem to enjoy tinkering with networks and say they engage in the project because of the technical challenge.
%TODO: compare to foss: enjoyable activity, intrinsic motivation

%enhance technical knowledge/skills
Some confessed that they wanted to polish their technical skills and acquire deeper understanding of the workings of (wifi mesh) networks, however this was also not the main motive for them to be active in the Freifunk community.
%TODO: compare to foss

%net neutrality
Although net neutrality is one of the key principles in the Pico-Peering-Agreement, the minimal consensus paper/document(?) for all Freifunk communities/free network projects, not all of the survey participants interpreted the notion the same way, nor did they grant it equal importance.

%initial motivation: personal need for fast internet(before) vs political ideas (now)
The initial motivations for joining the project for the different generations \textit{Freifunkers} become visible:
for people who joined before 2008 one of the main concerns was lack of fast internet connections in their area of living.
In contrast, those who started contributing after that were mainly (syn!) motivated by interest in the technical aspects of the project or its political idea/aspirations.
%TODO: compare to foss: there are also people who started to contribute to a project because of a personal need (``I needed feature XY'')

The possible motivations were summarised by one of the participants in the following manner: ``there are the ego people and there are the altruists''.

On the whole, we note following similarities and contrasts among motivations in the Freifunk Community compared to FOSS.
%TODO: Probably split this between the lines to acompany every observation

% monetary compensation
None of the activists engages in the Freifunk project because of a monetary compensation.

% better job opportunities/networking
Neither are networking or getting better job opportunities a driving force for the participants.
Some of them talked of these as byproducts of their engagement.
However, they underlined that they had not joined the project in order to become visible for potential employers.
%TODO: compare to foss! there should be people who participate out of this motive


%-------------------
%--bottom line
%-------------------

Although I'm still not quite sure how to measure altruism, and a comparison is really difficult here due to the radically different methodologies and participants' samples, I dare say that nowadays the prevailing reason for people to engage into the Freifunk project is their fascination with its ideology rather than other, more extrinsic motives.

\begin{table}[h]
%  \centering
  \begin{tabular}{| r | p{\textwidth} |}
    \hline
    \multicolumn{2}{|p{\textwidth}|}{Freifunk is technically challenging and it's fun to work on the project} \\
    \hline
    10 & yes \\
    1  & it's not thaat challenging, but it's fun \\
    1  & yeah\ldots it's not the primary motivation; it's challenging but not always fun \\
    1  & the technical part is rather boring \\
    \hline
  \end{tabular}
\end{table}

\begin{table}[h]
%  \centering
  \begin{tabular}{| r | p{\textwidth} |}
    \hline
    \multicolumn{2}{|p{\textwidth}|}{I want to learn the technical aspects of how wifi/mesh networks/etc. work} \\
    \hline
    6 & yes \\
    1 & I had to, in order to implement the concept \\
    1 & that wasn't my intention, but it happened at the end: ``x does't work, I want to fix it'' \\
    2 & yeah, I wanted to learn individual aspects, but it wasn't my primary focus \\
    3 & basically, I already knew how everything worked \\
    \hline
  \end{tabular}
\end{table}

\begin{table}[h]
%  \centering
  \begin{tabular}{| r | p{\textwidth} |}
    \hline
    \multicolumn{2}{|p{\textwidth}|}{I wanted to learn more about the legal framework within which the project operates} \\
    \hline
    9 & well, it's inevitable if you want to make the project work (not very enjoyable but important) \\
    2 & noe, not really \\
    1 & noe, I don't want to fit into predifined hierarchies \\
    1 & it wasn't a motivation to join Freifunk, but it's interesting to deal with it \\
    1 & it isn't uninteresting, but it's better if it's dealt with by people, who can do it better (who understand the legal formulations?) \\
    1 & yeah (works on the lobby field, is politically motivated; not initial motivation, but it later became one) \\
    \hline
  \end{tabular}
\end{table}

\begin{table}[h]
%  \centering
  \begin{tabular}{| r | p{\textwidth} |}
    \hline
    \multicolumn{2}{|p{\textwidth}|}{I want to have a free communication infrastructure in the hands of the community, because I think it's not a good idea that this is controlled by a couple of business players} \\
    \hline
    8 & yes (3 of them affirmed pretty vigorously) \\
    2 & yes (communication infrastructure in the hands of civil society; the industry has its own, the state too) \\
    3 & yes (but I don't believe anymore that Freifunk can provide this completely) \\
    1 & yes (note: it's difficult, if a lot of the decentralised infrastrucutre belongs to a single person) \\
    \hline
  \end{tabular}
\end{table}

\begin{table}[h]
%  \centering
  \begin{tabular}{| r | p{\textwidth} |}
    \hline
    \multicolumn{2}{|p{\textwidth}|}{I started to participate because commercial providers didn't have any interest of providing me with an efficient Internet connection} \\
    \hline
    2  & yes \\
    10 & no \\
    1  & no, but I'll be shortly working on a project for a refugee shelter, where this is the case \\
    1  & no, via Freifunk I share my Internet connection with others \\
    \hline
  \end{tabular}
\end{table}

\begin{table}[h]
%  \centering
  \begin{tabular}{| r | p{\textwidth} |}
    \hline
    \multicolumn{2}{|p{\textwidth}|}{I care for net neutrality} \\
    \hline
    6 & yes \\
    1 & yes (but sensibility for this issue is not necessarily present in the Freifunk context) \\
    4 & yes (but doesn't regard Freifunk as a struggle for net neutrality) \\
    1 & yes (but to demand it at any price what is difficult and not particularly clever) \\
    1 & yes (the Pico Peering Agreement is really important!) \\
    1 & yes (one can at least try\ldots) \\
    \hline
  \end{tabular}
\end{table}

\begin{table}[h]
%  \centering
  \begin{tabular}{| r | p{\textwidth} |}
    \hline
    \multicolumn{2}{|p{\textwidth}|}{I'm providing professional services related to Freifunk: I receive money for hardware installations, maintenance, etc} \\
    \hline
    9 & no \\
    1 & yeah, one can say there's something professional\ldots I invest a lot of time and energy in the project and have on occasions received money for my work in this context \\
    1 & I work professionally with wifi and mesh (not Freifunk); it was more a consequence of Freifunk than a motivation for collaboration \\
    1 & I work professionally on wifi installations but try to keep both things separate \\
    \hline
  \end{tabular}
\end{table}

\begin{table}[h]
%  \centering
  \begin{tabular}{| r | p{\textwidth} |}
    \hline
    \multicolumn{2}{|p{\textwidth}|}{I want to work in this field later, so my participation in Freifunk is an excellent opportunity to polish my technical skills, make new contacts and become visible for potential employers} \\
    \hline
    2 & no (it was more the other way round: that was a field where I could apply the knowledge I already had) \\
    6 & no \\
    1 & no (but if I wanted to do wifi professionally, the Freifunk community would be a good networking platform) \\
    3 & no (it surely happens, but it isn't a motivation for me to engange in the project) \\
    2 & no (it's something I'd write on my CV, but it wasn't a reason for me to join) \\
    \hline
  \end{tabular}
\end{table}


\begin{comment}
# Die Konkreten Fragen

### Freifunk ist technisch spannend/anspruchsvoll und es macht Spaß, sich damit auseinander zu setzen

* 10 ja
* 1 nicht soo anspruchsvoll aber schon spannend
* 1 joa, nicht Primärmotivation: technisch anspruchsvoll, aber nicht immer spannend^^;
* 1 technisch eher langweilig

---

### Ich will mir die technischen Fähigkeiten aneingnen, wie WLAN/Mesh-Netzwerke/... funktionieren

* 6 ja
* 1 muss man zwangsläufig, um die idee umzusetzen
* 1 nicht mit der Absicht hingegangen aber kam später dazu; "xy läuft nicht, ich wills fixen"
* 2 ja, einzelne Themen, nicht Primärfokus
* 3 Vorwissen mitgebracht/wusste schon wie alles geht im Grunde

---

### Ich will mich mit den gesetzlichen Rahmenbedinungen auseinandersetzen

* 9 naja, muss man zwangsläufig, wenn man das Projekt betreiben möchte (spannend unklar, aber wichtig)
* 2 nee eher nicht
* 1 nee, will sich nicht in vorgegebenen hierarchischen Strukturen einfügen
* 1 keine Einstiegsmotivation, aber spannend sich damit auseinander zu setzen
* 1 ist nicht uninteressant, sollten sich aber lieber Menschen mit beschäftigen, die das besser können (die juristische Sprache verstehen?)
* 1 ja (im Lobbyumfeld unterwegs, politische Motivations dabei; nicht Hauptmotivation, aber kam später dazu)

---

### Ich will eine freie, community-betriebene Kommunikationsinfrastruktur (mit)schaffen, weil ich denke, es ist Scheiße, dass alles in den Händen von paar Konzernen ist

* 8 ja (3 davon sehr stark bejaht :))
* 2 ja (zivilgesellschaftliche Kommunikationsinfrastruktur; die Industrie hat eine eigene; der Staat auch)
* 3 ja (aber glaub nicht mehr, dass das Freifunk komplett bringen kann)
* 1 ja (ist schwierig wenn einer Person viel von der dezentralen Infrastruktur gehört)

---

### Ich hab damit angefangen, weil die kommerziellen Providern keine Lust hatten, mich anzuschließen, weil es sich für sie nicht gelohnt hat

* 2 ja
* 10 nein
* 1 nein, aber demnächst an einem geflüchtetenprojekt mitarbeiten, wo das durchaus motivation ist
* 1 nein, stellt anderen den eigenen Anschluss zur Verfügung

---

### Mir ist Netzneutralität wichtig

* 6 ja
* 1 ja (sesibilität dafür im Freifunk kontext ist nicht unbedingt gegeben)
* 4 ja (aber sieht FF nicht als Kampf für Netzneutralität)
* 1 ja (aber die um jeden preis zu fordern ist schwierig und nicht sehr schlau)
* 1 ja (Pico-Peering-Agreement ist total wichtig!)
* 1 ja (man kanns zumindest versuchen :))

---

### Ich mach das mit Freifunk beruflich: krieg Geld für Knoten einrichten, Wartung, ...

* 9 nein
* 1 joa, kann man als Beruf sehen^^ steckt ziemlich viel Zeit und Energie rein und hat schonmal im Kontext Geld für die eigene Arbeit bekommen
* 1 macht jetzt was mit WiFi und Mesh beruflich (nicht mit Freifunk); eher Konsequenz aus Freifunk als Motivation dafür
* 1 macht beruflich was mit wlan installationen, versuchts aber von ff zu trennen

---

### Ich mach/will machen was auf dem Gebiet beruflich und das ist die Gelegenheit, meine Fähigkeiten aufzupollieren/Kontakte knüpfen/für Arbeitgeber_innen sichtbar werden

* 2 nein (eher: war der Bereich, wo ich meine bereits vorhandenen Kenntnisse anwenden konnte)
* 6 nein
* 1 nein (aber wenn was mit netzwerk machen wollen würde, würde die ff community schon als Kommunikations-/Kontaktplattform sehen)
* 3 nein (passiert sicher, ist aber keine motivation sich mit dem Projekt zu beschäftigen)
* 2 nein (schreibt man schon in den Lebenslauf aber keine Motivation um dahinzugehen)
\end{comment}


