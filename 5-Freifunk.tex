\section{Motivations in the Freifunk Community}
%TODO: Rename section to "Findings"?

general part of the questionnaire
--> cluster answers

some graphs/tables on the specific questions

\begin{comment}
* is the focus of the paper;

* only pull up the foss motivations as comparison to underline similarities and differences with the foundings here

* find a suitable neat form for presenting the foundings
\end{comment}

\begin{comment}
  ## Spannend (Manche zusammengefasste Beobachtungen):

* Nicht alle verstehen das selbe unter Netzneutralität und nicht alle erachten das als wichtig (je nach Verständnis)
* (initielle) Mitmachmotivationen verschiedener Generationen werden sichtbar:
  ** Menschen, die vor 2008 angefangen haben: Probleme mit (schnellem) Internetzugang bekommen, man nimmt das in die eigenen Hände
  ** Menschen, die später angefangen haben: eher aus Interesse an Netzwerktechnik/ finden die polit. Idee gut
* mehrerer sprechen den Community-Aspect an: also die Möglichkeit gemeinsam ein Projekt aufzubauen, interessante Menschen kennen zu lernen, "aus dem eigenen Dunst rauskommen"
* nicht viele Menschen sprechen die Idee an, das eigene Wissen zu teilen (auch wenn es eine der Zentralthemen in der Selbstdarstellung vom Projekt ist)
* "es gibt die Ego-Leute und es gibt die Altruisten"


##  Gemeinsamkeiten/Unterschiede in den Motivationen

* niemand macht Freifunk, weil er_sie dafür bezahlt wird
* (fast) niemand nimmt am Projekt teil, weil er_sie sich davon bessere Kontakte/Jobchancen verspricht (wird eher als Nebenprodukt gesehen) (an FOSS dagegen angeblich schon)
* sowohl bei FOSS als auch bei Freifunk gibt es Menschen, die aus persönlichen Bedürfnissen teilnehmen ("Ich hatte kein Netz" / "Ich brauchte das Feature XY")
* auch wenn mir unklar ist, wie ich Altruismus messen sollte, scheinen bei Freifunk die Mehrheit der Menschen von der politischen Idee einer dezentralen autonomen Kommunikationsinfrastruktur bewegt zu werden
* Communitybildung/-zugehörigkeit wichtig
* ein Großteil der Menschen findet die Beschäftigung mit Netzwerken spannend
* einige wollen ihre Netzwerkkenntnisse und -fähigkeiten ausbauen, scheint aber nicht so in Fokus zu sein (dagegen bei FOSS eher)

---

Die Papers beschäftigen sich (vor allem) mit der Motivation von Entwickler*innen. Allerdings gibt es auch andere Möglichkeiten zu beiden Communities beizutragen (user documentation, publicity, design, Übersetzung, ...). Kann man sich da andere Motivationen vorstellen?
\end{comment}

