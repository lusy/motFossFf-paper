\section{Motivations in the Freifunk Community}
%TODO: Rename section to "Findings"?

\begin{comment}
\begin{itemize}
  \item is the focus of the paper;
  \item only pull up the foss motivations as comparison to underline similarities and differences with the foundings here
  \item general part of the questionnaire $\rightarrow$ cluster answers
  \item some graphs/tables on the specific questions
\end{itemize}
\end{comment}

Leaning on the intrinsic--extrinsic motivation continuum outlined in Section~\ref{subsec:motivation} the reasons for the interviewees to engage in the Freifunk project can be loosely organised in the following clusters.

%\subsection{Intrinsic Motivation}
\textbf{Intrinsic Motivation}
%-- it's fun to climb roofs and get to see the city from above; to help develop the firmware; ...
%-- interest, enjoyment, inherent satisfaction

%technical challenge/fun (intrinsic)
As explained, intrinsic motivation denominates the willingness of people to engage in activities they find interesting and enjoyable by itself.
Most participants in the interviews seem to enjoy tinkering with networks and say they engage in the project because of the technical challenge.
An activist mentioned that he found the Freifunk project more tangible and consequently more enjoyable than software development.
Another one spoke with eagerness of the challenge to produce maximal results with minimal resources.
Yet another referred enthusiastically to the otherwise scarce opportunities to climb on church and townhall towers and to enjoy the view both inside the old buildings and over the rooftops of the city during antenna installations.

% project is inherently exciting (intrinsic)
The wish to do interesting and meaningful things in one's leisure time was also stated as a motive to turn to Freifunk.
The participants looked for a project where they could apply their knowledge and experience in their own creative manner without the demands of hierarchies and bosses.
The size of the project, the diversity of tasks involved and consequently the possibility to engage in different activities was also appealing to the activists.

%compare to foss: enjoyable activity, intrinsic motivation
The interest in the activity at hand and inherent satisfaction is an important and often named motivation for FOSS developers, although both consulted surveys on FOSS motivation found that it was not the primary reason for people to contribute~\cite{HarOu2002},~\cite{LakWo2005}.
As sketched in Section~\ref{subsec:comparison}, we can also assert that the range of possible tasks in FOSS development is not as diverse as in a community network project.

%\subsection{Extrinsic Motivation: political ideology}
\textbf{Extrinsic Motivation: (political) ideology}

According to Ryan and Deci, if individuals act out of conviction, because they identify with a certain set of values, we are presented with external motivation, but one where self-determination is strong and consequently the motivation itself is it as well~\cite{RyDe2000}.

%FOSS
It appears that identification with the FOSS ideology and community is indeed important for some participants in FOSS projects, although not to the extend one might have expected (both investigations find this to be driving force for $~1/3$ of their participants)~\cite{HarOu2002},~\cite{LakWo2005}.
% political motivation (extrinsic? values, community)
For the majority of the interviewees in the Freifunk survey however, the political aspect of the project seemed to play a very central role.
Many of them mentioned on their own that the idea of a decentralised, non-hierarchical and non-commercial communication was one of the fundamental motives which drove them to engage in the project in the first place.
They spoke of ``a right to free communication and information'' and sovereignty which can only be truely granted if people build their own infrastructureand organise its operation in such a manner that no single person is able to shut it down.
When prompted by one of the specific questions (see Appendix) all participants agreed that it was important to build a free communication infrastructure controlled by civil society and not by the state or influential business players.

At the same time, several activists expressed their regrets that unfortunately the vision didn't scale technically as well as their concern that it was not easy to explain and propagate these ideas outside of the community: people didn't always seem to understand or care for informational self-determination.
It was not until the media started to cover Freifunk's engagement in connecting refugee shelters throughout Germany to the Internet, that the concept seemed to become somewhat clearer for mainstream society.
Moreover, community members complained of the service mentality of some users who appeared to view Freifunk as yet another service provider and not to understand its essence as an emancipatory hands-on project, which after a period of time tended to drive contributors away.
Then again, there were also participants who feared that the Freifunk community didn't try hard enough to engage and be open to folks with non-technical background.

\begin{comment}
%net neutrality (extrinsic, community, ideology)
Although net neutrality is one of the key principles in the Pico-Peering-Agreement, the minimal consensus paper/document(?) for all Freifunk communities/free network projects, not all of the survey participants interpreted the notion the same way, nor did they grant it equal importance.
\end{comment}


%\subsection{Extrinsic Motivation: feeling part of a community/community obligation}
\textbf{Extrinsic Motivation: feeling part of a community/community obligation}
%Working for free... classifies that as Intrinsic motivation
%Why hackers do... too

% feeling part of a community
Several participants mentioned the community aspect of their work, the feeling of belonging to a community they admired, as a driving force.
They talked about ``building a project together with others''\footnote{The interviews were conducted in German. Here mentioned citations have been translated by the author. The author carries responsibility and apologises for any inaccuracies.}
, ``collaborating with and getting to know people of different ages and backgrounds, which would have hardly happened in a different setting'', ``expanding one's horizons and getting out of one's comfort zone''.
We can recognise here the ``relatedness to others'' component from the motivations' research which apparently drives individuals to internalising the activity they engage in.
%TODO: compare with foss
% * "technisch interessant --> Schnittstelle zwischen Community und Technik"
Some found the intersection of community and technic was the most interesting part of the project.
Finding out how it works (or does not work) to organise a community, what volunteer work means, what people do with pleasure and which tasks get ignored and forgotten and why, who has the power to decide things and whether it is necessary to debate and formulate decissions for everything, and most importantly, how to involve new participants so that the project does not fall apart and how to prevent and deconstruct hierarchies in knowledge are only few of the interesting questions with which (community) activists have to deal.
One of the activists, who has been working on the project from the very beginning, spoke with enthusiasm about how big the community had become.

% sharing knowledge (values, community) + criticism that it is not enough
Curiously, only two(?) participants viewed sharing knowledge, empowering others to build their own infrastructure and educating people about the setup and workings of (mesh-)networks, as well as aspects such as their inherent secury, privacy and neutrality as motivation, although these are among the central goals sketched by the community in their self-conception\cite{ffweb}.
What is more, these same individuals expressed their concern that these ideas didn't seem to be addressed sufficiently.
They criticised the fact(syn?) that despite the ideological intention to maintain horizontal structures, people(syn) did tend to eventually build up (knowledge) hierarchies into the project and that in some cases decisions/events were driven more by the egos of particular participants rather than by the self-proclaimed principles of the project.
It was also suggested/remarked that although unintentionally, due to knowledge distribution and readiness to invest(?) time in the project, often it were only few people who ended up taking care of major part of the network which both contradicted the principles and led to significant workloads and overloads for these individuals.

% an event gave the initial impulse
Many of the interviewed only started to actively participate in Freifunk after attending some kind of event: be it a local community meeting, meeting activists at the Chaos Communication Congress or other kinds of conferences.
%TODO: also part of community?? or in general: personal communication is what actually spurs people into action?


%\subsection{Extrinsic Motivation: Ego Involvement, expected approval from self or others}
\textbf{Extrinsic Motivation: Ego Involvement, expected approval from self or others}
%peer recognition
%Why hackers do... classify that as intrinsic motivation

%following 2 paragraphs together? or not?
% feeling that one's work is needed and cherished (more extrinsic, but to feel needed --> relatedness to others)
What is more, a couple of the activists mentioned that they were moved by the feeling that their work was useful and cherished by others.
The positive feedback they'd received from people using the open network (among others many students and refugees) inspired them to continue their engagement in the project.


%\subsection{External Motivation: Satisfying (personal) needs}
\textbf{External Motivation: Satisfying (personal) needs}
%code/result needed by employer or self

%initial motivation: personal need for fast internet(before) vs political ideas (now)
The initial motivations for joining the project for the different generations \textit{Freifunkers} become visible:
for people who joined before 2008 one of the main concerns was lack of fast internet connections in their area of living.
In contrast, those who started contributing after that were mainly (syn!) motivated by interest in the technical aspects of the project or its political idea/aspirations.
%TODO: combine with political ideas above
%TODO: compare to foss: there are also people who started to contribute to a project because of a personal need (``I needed feature XY'')
Although none of the recently joined activists mentioned personal need as a driving force,
some of them stated that it was important for them to use the project in order to connect others to the Internet or share their existing connections with people who needed it.
Several interviewees explained their participation in campaigns to supply refugee shelters throughout Germany with Internet connections was amongst their (primary) motivations to engage in Freifunk.
One participant asserted that this factor had actually brought new communities into being.


The possible motivations were summarised by one of the participants in the following manner: ``there are the ego people and there are the altruists''.


%\subsection{Extrinsic Motivation: expectance of a future reward}
\textbf{Extrinsic Motivation: expectance of a future reward}
%  polish skills,
%  networking/self-marketing
%  selling product

%enhance technical knowledge/skills (extrinsic, but self-determined)
Some confessed that they wanted to polish their technical skills and acquire deeper understanding of the workings of (wifi mesh) networks, which is ``hardly possible to this extent in another project which one can do in private in one's spare time''.
However this was also not the main motive for them to be active in the Freifunk community.
%TODO: compare to foss

% monetary compensation (extrinsic)
None of the activists engages in the Freifunk project because of a monetary compensation.

% better job opportunities/networking (extrinsic)
Neither are networking or getting better job opportunities a driving force for the participants.
Some of them talked of these as byproducts of their engagement.
However, they underlined that they had not joined the project in order to become visible for potential employers.
%TODO: compare to foss! there should be people who participate out of this motive




%-------------------
%--bottom line
%-------------------

Although I'm still not quite sure how to measure altruism, and a comparison is really difficult here due to the radically different methodologies and participants' samples, I dare say that nowadays the prevailing reason for people to engage into the Freifunk project is their fascination with its ideology rather than other, more extrinsic motives.



%----------------------------------------------
On the whole, we note following similarities and contrasts among motivations in the Freifunk Community compared to FOSS.
%TODO: Probably split this between the lines to acompany every observation

\begin{comment}
# Was ist am Projekt nicht so cool?

### Kommunikations/Diskussionskultur

* "Kommunikationskultur"
* "Arbeit im Team ist manchmal schwierig"
* "wenn Leute meinen, den richtigen Weg gefunden zu haben und den anderen aufzuzwingen und alles an sich reißen"
* "wenn sich Menschen richtig bekämpfen, hat das das Potenzial auch das Gesamtprojekt richtig runter zu ziehen"
* "treffen, die man viel redet und nichts passiert sind auch lästig"
* "leute, die in die presse sich als organisator_innen ausgeben und nur auf dem ruhm aus sind; trotzdem ist auch wichtig, dass jemand mit der presse und mit der stadt redet"

---

### Gender

* "genderthemen: community besteht größtenteils aus männern, die zu öffnen/erweiter funktioniert bis jetzt nicht so gut"

---
\end{comment}
