\section{Basics}
\subsection{FOSS}
All of us are familiar with the characteristics of the Free and Open Source Software and the corresponding community.
To sum it up briefly, these are centered around the idea that source code should be open and freely accessible.
What is more, according to the Free Software Movement,
some fundamental freedoms such as running a program for any purpose, distributing copies of it, modifying the code and distributing copies of the modified version should be granted. (cite!)


\subsection{Community networks \& the Freifunk project}
Slightly less clear is perhaps what is meant by the term ``free/community networks''. %TODO: clarify terms free networks? community networks? using it synonimously?
%We'll concentrate on the definition by..
These constitute a free, decentralised communication infrastructure built and controlled by civil society.
Many projects also involve the development of free software -- the router firmware, although that is not an inherent characteristic of all of them.
Further specifics/essentials/key points are the free access to the network for anybody with a wifi capable/enabled device and the transfer of data within the network without their/its tampering/.. or prioritizing. (cite!)

Freifunk is one particular example of a free community network project.
It was selected as primary focus for the current research for one obvious reason: it consists of local communities spread throughout Germany, so the author had a direct contact with the activists.
Although there are an aspiration for cooperation and some degree of coordination, the single communities are to great extent autonomous, each of them organized in its own manner.
Guiding principles to which all Freifunk communities are committed are...
%TODO: cite: freifunk website? selbstverständnis?

%TODO: separate section?
A minute of contemplation reveals some similarities but also some key differences between both domains and their associated/corresponding activities.
Whereas the work of the FOSS community concentrates around a digital artefact, the work of free network activists extends well beyond that.
Some community network projects, among them Freifunk, develop their own free firmware (based on OpenWRT?), but this is only one of the activities in which Freifunkers engage.
Further tasks include network maintenance (so continuous commitment is required in contrast to foss where one can write a piece of software, release it to the general public and basically forget about it (if not developing new features/fixing bugs)) and on-site installation (we note: hardware costs are involved), as well as spreading the idea and convincing new people to participate in the project and extend the network, grant the project access to key locations for antenna installation (high buildings, rooftops) or contribute some funds.

\subsection{Motivation}
> To be motivated means to be moved to do something. (cite?)
According to well-established psychological research on the field, motivation can be roughly classified into 2 cathegories:
intrinsic, which denominates the impulse to engage in an activity that is by itself interesting and entertaining, and extrinsic, where some kind of external punishment or reward is involved.
Often, the second type of motivation is presented as less valid and producing poorer results.
However, extrinsic motivation is hardly a homogeneous cathegory, but should instead be viewed as a continuum, where productivity and satisfaction expand with increasing feelings of competence, autonomy and relatedness to others. (cite! Ryan and Deci 2000)

\begin{comment}
What is FOSS (really brief, people should know that)
What is Freifunk (slightly less brief, is the new community) <--- that's the focus

Comparison of the domains:
* FOSS: digital products
* community networks:
  ** digital products (the firmware)
  ** hardware costs
  ** network maintanance
  ** network installation on site
  ** talk to people and convince them of the idea in order to
     *** get access to suitable locations
     *** get funds
     *** convince them to install their own routers and extend the network -- ``marketing'' aspect a way more pronounced

What types of motivation are relevant
\end{comment}
