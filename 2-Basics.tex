\section{Basics}
\subsection{FLOSS}
Free, Libre and Open Source Software and the corresponding community are centered around the idea that source code should be open and freely accessible.
What is more, according to the Free Software Movement,
some fundamental freedoms such as studying the way a program works, running it for any purpose, distributing copies of it, modifying the code and distributing copies of the modified version should be granted\cite{gnuweb}.


\subsection{Community networks \& the Freifunk project}
Slightly less clear is perhaps what is meant by the term ``community networks''.
%We'll concentrate on the definition by..
These constitute a free, decentralised communication infrastructure built and controlled by civil society.
Since the hardware is not in the hands of the state or any business players, they cannot exercise any censure over the contents spread over it. %control/censure
The majority of the community network projects build upon wireless technologies because of their cheap cost and the permissive legal regulations\cite{WNDW2013}\cite{Medosch2004}.
Many of them aim for creating mesh networks where the data is routed in a decentralised manner and direct data exchange between immediate neighbors is possible.
This architecture brings reliability and redundancy, allowing for robust bottom-up networks\cite{Medosch2004}.
A lot of projects also involve the development of free software---the router firmware, although that is not an inherent characteristic of all of them.
Further specifics are the free and anonymous access to the network for anybody with a wifi capable device and the transfer of data within the network without its tampering, inspection or prioritising\cite{ffweb}\cite{Medosch2004}. %specifics/essentials/key points

Freifunk is one particular example of a community network project.
It was selected as primary focus for the current research because it consists of local communities spread throughout Germany, so the author had a direct contact with the activists.
Although there are an aspiration for cooperation and some degree of coordination, the single communities are to great extent autonomous, each of them organised in its own manner.
Apart from the political idea of a decentralised mesh network owned by its users and maintainers, guiding principles to which all Freifunk communities are committed are: reducing the digital divide, empowering people and creating awareness on communication and freedom of information\cite{ffweb}.

\subsection{Community networks and FLOSS in comparison}\label{subsec:comparison}
There are some similarities but also some key differences between both domains and their corresponding activities. %associated/corresponding
Whereas the work of the FLOSS community concentrates around a digital artefact, the work of community network activists extends well beyond that.
Some community network projects, among them Freifunk, develop their own free firmware (most of them based on OpenWRT), but this is only one of the activities in which Freifunkers engage.
Further tasks include:
\begin{itemize}
  \item on-site installation (we note: hardware costs are involved),
  \item network maintenance,
  \item as well as spreading the idea and convincing new people to participate and extend the network, grant the project access to key locations for antenna installation (high buildings, rooftops) or contribute funds. %funds/financial support
\end{itemize}

\subsection{Motivation}\label{subsec:motivation}
``To be motivated means to be moved to do something''~\cite{RyDe2000}.
According to well-established psychological research, motivation can be roughly classified into two categories:
intrinsic, which denominates the impulse to engage in an activity that is by itself interesting and/or entertaining,
and extrinsic, where some kind of external punishment or reward is involved.
Often, the second type of motivation is presented as less valid and producing poorer results. %(impoverished and less effective)
However, extrinsic motivation is hardly a homogeneous category, but should instead be viewed as a continuum where productivity and satisfaction expand with increasing feelings of competence, autonomy and relatedness to others~\cite{RyDe2000}.
The psychologists Ryan and Deci open several subcategories within the extrinsic motivation ranging from acting for the sake of an expected future reward over expecting an approval from self or others to consciously valuing an activity and identification with a community and its goals~\cite{RyDe2000}.
The motivations of participants in the Freifunk community network project are investigated within this framework and compared to some of the results of the surveys on motivation in FLOSS conducted by Lakhani and Wolf in 2005~\cite{LakWo2005} and Hars and Ou in 2002~\cite{HarOu2002}.

\begin{comment}
What is FLOSS (really brief, people should know that)
What is Freifunk (slightly less brief, is the new community) <--- that's the focus

Comparison of the domains:
* FLOSS: digital products
* community networks:
  ** digital products (the firmware)
  ** hardware costs
  ** network maintanance
  ** network installation on site
  ** talk to people and convince them of the idea in order to
     *** get access to suitable locations
     *** get funds
     *** convince them to install their own routers and extend the network -- ``marketing'' aspect a way more pronounced

What types of motivation are relevant
\end{comment}
