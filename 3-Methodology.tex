\section{Methodology}

Due to the gap in the scientific literature on motivation in community network projects, primary field research in the form of semi-structured interviews was conducted.
The questions for the interviews were inspired by the questionnaires applied by scientists investigating the FOSS motivations\cite{HarOu2002}\cite{LakWo2005} and complemented by questions the author considered relevant/insightful. %TODO: include questions if there's space
The interviews were conducted in person (all but one of them, which was conducted using a common voip software) and were not recorded and transcribed, but instead, foundation for the present paper form the extensive notes taken by the author. % is "I" frowned upon?
The author is well aware of the limitations such approach poses, nevertheless, it was considered a quick and useful way to gain some first impressions on the wide variety of reasons which spur community network activists.

Sixteen interviews were conducted in that manner over a period of two months in the beginnings of 2016.
Some demographic information about the participants is summarised in table~\ref{tab:communities} and figure~\ref{fig:demography}.

\begin{table}[h]
%\centering %used for centering table
  \begin{tabular}{| l | r |}
    \hline
    \textbf{community} & \textbf{participants}\\
    \hline
    Berlin & 9 \\
    \hline
    Rheinland & 3 \\
    \hline
    Bielefeld & 1 \\
    \hline
    Bremen & 1 \\
    \hline
    Hamburg & 1 \\
    \hline
    Hannover & 1 \\
    \hline
  \end{tabular}
\caption{Survey participants according to their community}
\label{tab:communities}
\end{table}


\begin{figure}[h]
  \begin{subfigure}[h]{0.5\textwidth}
    \centering
    \includegraphics[scale=0.3]{diagrams/gender_py.png}
    \caption{Perceived participants' gender}
    \label{fig:gender}
  \end{subfigure}
  \qquad
  \begin{subfigure}[h]{0.5\textwidth}
    \centering
    \includegraphics[scale=0.3]{diagrams/length_particip_num.png}
    \caption{Length of involvement}
    \label{fig:length}
  \end{subfigure}
  \caption{Participants' demographics}
  \label{fig:demography}
\end{figure}

%* demographics (gender/community)
%** pie chart/histogram with gender
%** table with community?
%** pie chart how long have people participated

\begin{comment}
\begin{itemize}
  \item literature review for FOSS <-- only as comparison
  \item semi-structured interviews for Freifunk <-- focus
    \begin{itemize}
      \item limitations of the methodology
    \end{itemize}
\end{itemize}
\end{comment}
