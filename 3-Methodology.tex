\section{Methodology}

Due to the gap in the scientific literature on motivation in community network projects, primary field research in the form of semi-structured interviews was conducted.
The questions for the interviews were inspired by the questionnaires applied by scientists investigating the FLOSS motivations\cite{HarOu2002}\cite{LakWo2005} and complemented by questions the author considered relevant or insightful.
The interviews were conducted in person (all but one of them, which was conducted using a common VoIP software) and were not recorded and transcribed, but instead, foundation for the present paper form the extensive notes taken by the author.
Such approach poses certain limitations, nevertheless, it was considered a quick and useful way to gain some first impressions on the wide variety of reasons which spur community network activists.

Sixteen interviews were conducted in that manner over a period of two months in the beginnings of 2016.
The raw data gathered during the interviews can be consulted on github\cite{FFInterviews} (in German).
Some demographic information about the participants is summarised in Figure~\ref{fig:demography}.

\begin{figure}[h]
  \begin{subfigure}[h]{\textwidth}
    \centering
    \begin{tabular}{ l | r }
      \hline
      \textbf{community} & \textbf{members}\\
      \hline
      Berlin & 9 \\
      %\hline
      Rheinland & 3 \\
      %\hline
      Bielefeld & 1 \\
      %\hline
      Bremen & 1 \\
      %\hline
      Hamburg & 1 \\
      %\hline
      Hannover & 1 \\
      \hline
    \end{tabular}
  \caption{Survey participants according to their community}
  \label{tab:communities}
  \end{subfigure}
  \qquad
  \begin{subfigure}[h]{0.45\textwidth}
    %\centering
    \includegraphics[scale=0.3]{diagrams/gender_py.png}
    \caption{Perceived participants' gender}
    \label{fig:gender}
  \end{subfigure}
  \qquad
  \begin{subfigure}[h]{0.45\textwidth}
    \centering
    \includegraphics[scale=0.3]{diagrams/length_particip_num.png}
    \caption{Length of involvement}
    \label{fig:length}
  \end{subfigure}
  \caption{Participants' demographics}
  \label{fig:demography}
\end{figure}

The applied questionnaire consists of two parts: one more general, where the participants were expected to provide some background information and share their experiences and views on the project, and one more specific, where they were prompted to explain whether or not certain factors were part of their motivation to engage in Freifunk.
The answers are discussed in the following section.
All the questions as well as the answers to the specific part can be viewed in the appendix.

\subsection{Threats to validity}
Beside the above mentioned methodological limitations,
another critical issue that should be noted here are the publication dates of the FLOSS reference papers.
Both papers researching motivation in the FLOSS communities~\cite{HarOu2002}~\cite{LakWo2005} are over ten years old, which leaves us doubting to what extend their findings are still valid today.
However, the author was unable to find more recent investigations dealing precisely with the motivation of participants in the FLOSS community in the same fashion the chosen papers do.
Therefore, the comparison is carried out as described, still, we should bear in mind that a present survey may provide quite different results and that people's motives changes over time.

\begin{comment}
\begin{itemize}
  \item literature review for FLOSS <-- only as comparison
  \item semi-structured interviews for Freifunk <-- focus
    \begin{itemize}
      \item limitations of the methodology
    \end{itemize}
\end{itemize}
\end{comment}
