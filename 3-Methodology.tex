\section{Methodology}

Due to the gap in the scientific literature on motivation in community network projects, primary field research in the form of semi-structured interviews was conducted.
The questions for the interviews were inspired by the questionnaires applied by scientists investigating the FLOSS motivations\cite{HarOu2002}\cite{LakWo2005} and complemented by questions the author considered relevant or insightful.
The interviews were conducted in person (all but one of them, which was conducted using a common voip software) and were not recorded and transcribed, but instead, foundation for the present paper form the extensive notes taken by the author.
The author is well aware of the limitations such approach poses, nevertheless, it was considered a quick and useful way to gain some first impressions on the wide variety of reasons which spur community network activists.

Sixteen interviews were conducted in that manner over a period of two months in the beginnings of 2016.
Some demographic information about the participants is summarised in Table~\ref{tab:communities} and Figure~\ref{fig:demography}.

\begin{table}[h]
%\centering %used for centering table
  \begin{tabular}{| l | r |}
    \hline
    \textbf{community} & \textbf{participants}\\
    \hline
    Berlin & 9 \\
    \hline
    Rheinland & 3 \\
    \hline
    Bielefeld & 1 \\
    \hline
    Bremen & 1 \\
    \hline
    Hamburg & 1 \\
    \hline
    Hannover & 1 \\
    \hline
  \end{tabular}
\caption{Survey participants according to their community}
\label{tab:communities}
\end{table}


\begin{figure}[h]
  \begin{subfigure}[h]{0.5\textwidth}
    \centering
    \includegraphics[scale=0.3]{diagrams/gender_py.png}
    \caption{Perceived participants' gender}
    \label{fig:gender}
  \end{subfigure}
  \qquad
  \begin{subfigure}[h]{0.5\textwidth}
    \centering
    \includegraphics[scale=0.3]{diagrams/length_particip_num.png}
    \caption{Length of involvement}
    \label{fig:length}
  \end{subfigure}
  \caption{Participants' demographics}
  \label{fig:demography}
\end{figure}

The applied questionnaire consists of two parts: one more general, where the participants were expected to provide some background information and share their experiences and views on the project, and one more specific, where they were prompted to explain whether or not certain factors were part of their motivation to engage in Freifunk.
The answers are discussed in the following section.
All the questions (not true for short paper!) as well as the answers to the specific part can be viewed in the appendix.


\begin{comment}
\begin{itemize}
  \item literature review for FLOSS <-- only as comparison
  \item semi-structured interviews for Freifunk <-- focus
    \begin{itemize}
      \item limitations of the methodology
    \end{itemize}
\end{itemize}
\end{comment}
