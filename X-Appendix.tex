\section*{Appendix}

\subsection*{The questionnaire}

\begin{enumerate}[label*=\arabic*.]
  \item \textbf{General Part}
% \begin{itemize}
 \begin{enumerate}[label*=\arabic*.]
   \item Why/How have you engaged in the project for the first time?
   \item How long have you been participating?
   \item What exactly are you doing/have you done within Freifunk?
   \item Are you actively participating in organising the community work: meetings, etc.?
   \item Which aspects of the project are cool? Which are not so cool?
   \item What is your background?
% \end{itemize}
 \end{enumerate}

\item \textbf{More specific:} Are following factors of significance for your motivation?
%\begin{itemize}
\begin{enumerate}[label*=\arabic*.]
  \item Freifunk is technically challenging and it's fun to work on the project.
  \item I want to learn the technical aspects of how wifi/mesh networks/etc. work.
  \item I wanted to learn more about the legal framework within which the project operates.
  \item I want to have a free communication infrastructure in the hands of the community, because I think it's not a good idea that this is controlled by a couple of business players.
  \item I started to participate because commercial providers didn't have any interest of providing me with an efficient Internet connection.
  \item I care for net neutrality.
  \item I'm providing professional services related to Freifunk: I receive money for hardware installations, maintenance, etc.
  \item I want to work in this field later, so my participation in Freifunk is an excellent opportunity to polish my technical skills, make new contacts and become visible for potential employers.
%\end{itemize}
\end{enumerate}

\end{enumerate}

\subsection*{Answers to the specific questions}

  \begin{tabular}{| r | p{\textwidth} |}
    \hline
    \multicolumn{2}{|p{\textwidth}|}{Freifunk is technically challenging and it's fun to work on the project} \\
    \hline
    10 & yes \\
    1  & it's not thaat challenging, but it's fun \\
    1  & yeah\ldots it's not the primary motivation; it's challenging but not always fun \\
    1  & the technical part is rather boring \\
    \hline
  \end{tabular}

  \begin{tabular}{| r | p{\textwidth} |}
    \hline
    \multicolumn{2}{|p{\textwidth}|}{I want to learn the technical aspects of how wifi/mesh networks/etc. work} \\
    \hline
    6 & yes \\
    1 & I had to, in order to implement the concept \\
    1 & that wasn't my intention, but it happened at the end: ``x doesn't work, I want to fix it'' \\
    2 & yeah, I wanted to learn individual aspects, but it wasn't my primary focus \\
    3 & basically, I already knew how everything worked \\
    \hline
  \end{tabular}

  \begin{tabular}{| r | p{\textwidth} |}
    \hline
    \multicolumn{2}{|p{\textwidth}|}{I wanted to learn more about the legal framework within which the project operates} \\
    \hline
    9 & well, it's inevitable if you want to make the project work (not very enjoyable but important) \\
    2 & noe, not really \\
    1 & noe, I don't want to fit into predefined hierarchies \\
    1 & it wasn't a motivation to join Freifunk, but it's interesting to deal with it \\
    1 & it isn't uninteresting, but it's better if it's dealt with by people, who can do it better (who understand the legal formulations?) \\
    1 & yeah (works on the lobby field, is politically motivated; not initial motivation, but it later became one) \\
    \hline
  \end{tabular}

  \begin{tabular}{| r | p{\textwidth} |}
    \hline
    \multicolumn{2}{|p{\textwidth}|}{I want to have a free communication infrastructure in the hands of the community, because I think it's not a good idea that this is controlled by a couple of business players} \\
    \hline
    8 & yes (3 of them affirmed pretty vigorously) \\
    2 & yes (communication infrastructure in the hands of civil society; the industry has its own, the state too) \\
    3 & yes (but I don't believe anymore that Freifunk can provide this completely) \\
    1 & yes (note: it's difficult, if a lot of the decentralised infrastructure belongs to a single person) \\
    \hline
  \end{tabular}

  \begin{tabular}{| r | p{\textwidth} |}
    \hline
    \multicolumn{2}{|p{\textwidth}|}{I started to participate because commercial providers didn't have any interest of providing me with an efficient Internet connection} \\
    \hline
    2  & yes \\
    10 & no \\
    1  & no, but I'll be shortly working on a project for a refugee shelter, where this is the case \\
    1  & no, via Freifunk I share my Internet connection with others \\
    \hline
  \end{tabular}

  \begin{tabular}{| r | p{\textwidth} |}
    \hline
    \multicolumn{2}{|p{\textwidth}|}{I care for net neutrality} \\
    \hline
    6 & yes \\
    1 & yes (but sensibility for this issue is not necessarily present in the Freifunk context) \\
    4 & yes (but doesn't regard Freifunk as a struggle for net neutrality) \\
    1 & yes (but to demand it at any price what is difficult and not particularly clever) \\
    1 & yes (the Pico Peering Agreement is really important!) \\
    1 & yes (one can at least try\ldots) \\
    \hline
  \end{tabular}

  \begin{tabular}{| r | p{\textwidth} |}
    \hline
    \multicolumn{2}{|p{\textwidth}|}{I'm providing professional services related to Freifunk: I receive money for hardware installations, maintenance, etc} \\
    \hline
    9 & no \\
    1 & yeah, one can say there's something professional\ldots I invest a lot of time and energy in the project and have on occasions received money for my work in this context \\
    1 & I work professionally with wifi and mesh (not Freifunk); it was more a consequence of Freifunk than a motivation for collaboration \\
    1 & I work professionally on wifi installations but try to keep both things separate \\
    \hline
  \end{tabular}

  \begin{tabular}{| r | p{\textwidth} |}
    \hline
    \multicolumn{2}{|p{\textwidth}|}{I want to work in this field later, so my participation in Freifunk is an excellent opportunity to polish my technical skills, make new contacts and become visible for potential employers} \\
    \hline
    2 & no (it was more the other way round: that was a field where I could apply the knowledge I already had) \\
    6 & no \\
    1 & no (but if I wanted to do wifi professionally, the Freifunk community would be a good networking platform) \\
    3 & no (it surely happens, but it isn't a motivation for me to engage in the project) \\
    2 & no (it's something I'd write on my CV, but it wasn't a reason for me to join) \\
    \hline
  \end{tabular}


